\documentclass[titlepage]{article}
\usepackage{xcolor}
\usepackage{listings}
\usepackage{fullpage}
\lstset{language=[LaTeX]TeX,breaklines=true,basicstyle=\tt\normalsize,keywordstyle=\color{blue},identifierstyle=\color{magenta}}
%\lstset{language=[LaTeX]TeX,breaklines=true}
\title{\LaTeX \ for Theses --- Workbook}
\author{Dr Thomas Bishop \& Dr Gerard Capes}
\date{17\textsuperscript{th} November, 2016}
\begin{document}
\maketitle
\section{Opening TeXNicCentre}
\begin{enumerate}
	\item Navigate \texttt{Start > All Programs > Programming Languages > TeXNicCentre} and open the application. 
	\item You may need to configure the application on first run; accept the defaults. This uses the MiKTeX \LaTeX \ distribution that has already been installed.
	\item In the top menu ribbon, change the compiler from \texttt{\LaTeX \ > DVI} to \texttt{\LaTeX \ > PDF}.
\end{enumerate}
\section{Hello World!}
\begin{enumerate}
	\item Create a new document.
	\item In the writing pane, write the following lines: 		
\begin{lstlisting}
\documentclass{article}	
\begin{document}				
Hello World							
\end{document}					
\end{lstlisting}	
	\item Save the file.
	\item Click on the compile button.
	\item Open the pdf file.
\end{enumerate}	
\section{Customise the University Thesis Document}
For the following exercises, recompile the document after you have completed each set of changes
\begin{enumerate}	
	\item Find and modify the following lines	
	\begin{lstlisting}
\begin{document}
...
\title{The University Thesis File}
\author{The Author's name}
% Faculty of Life Sciences people should comment the next line out
\school{The Author's school}
\faculty{The Author's faculty}
	\end{lstlisting}
	\item Find and modify the contents of the abstract (located between \lstinline|\beforeabstract| and \lstinline|\afterabstract|)
	\item Add some acknowledgements in the section immediately after \lstinline|\prefacesection{Acknowledgements}|
\end{enumerate}

\section{Adding a Chapter}
	\begin{enumerate}
		\item Create a new document
		\item Type \lstinline|\chapter{Methodology}| in the writing pane to define it as the start of a chapter
		\item Try some different writing commands, using the examples below
% todo: Quote marks need fixing in this listing so they are not typeset, but display as would be seen in the editor
		\begin{lstlisting} 
This is going to include some ``quoted text''.				
Some characters are reserved (i.e. they have a special purpose) but can be accessed by escaping with a backslash, e.g. \%.
To use italics, try \textit{using this command!}.
Math mode is contained within dollars, e.g. $\delta$.
		\end{lstlisting}
		\item Include this chapter file in the main thesis.tex after the other included *.tex file. If you saved your chapter file as `mychapter.tex', use the command \lstinline|\include{mychapter}|.
		\item Recompile the thesis.tex file
	\end{enumerate}
\section{Referencing}
\section{Cross-referencing}
\section{Adding Figures}
\section{Drawing Tables}
\section{Writing Equations}
\section{Inserting a quote}
Try inserting a quote using the \lstinline!\quote! environment:
\begin{lstlisting}
\begin{quote}
I think therefore I use \LaTeX for typesetting. 
\end{quote}
\end{lstlisting}
\section{Escaped \& Special Characters}
Try using reserved characters like \&, \textbackslash, and \%, and you'll run into difficulty. Try writing the following:
\begin{lstlisting}
If you need to write an ampersand (\&), use the slash to escape or it will not render. 							
Similarly, if you wanted to use a slash or backslash (\backslash), you must use a command. If you need accents, they are often produced l\'ike that. 	
\end{lstlisting}
\end{document}