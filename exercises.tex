\documentclass[titlepage]{article}
\usepackage{textcomp}
\usepackage{xcolor}
\usepackage{listings}
\usepackage{fullpage}
\usepackage[type={CC},modifier={by-sa},version={4.0},]{doclicense}
\lstset{language=[LaTeX]TeX,breaklines=true,basicstyle=\tt\normalsize,keywordstyle=\color{blue},identifierstyle=\color{magenta}}

\title{\LaTeX \ for Theses --- Workbook}
\author{Dr Thomas Bishop \and Dr Gerard Capes}
\date{17\textsuperscript{th} of November, 2016}

\begin{document}

\maketitle

\section{Opening \TeX nicCentre}
	\begin{enumerate}
		\item Navigate \texttt{Start > All Programs > Programming Languages > TeXnicCentre} and open the application. 
		\item You may need to configure the application on first run; accept the defaults. This uses the MiKTeX \LaTeX \ distribution that has already been installed.
		\item In the top menu ribbon, change the compiler from \texttt{LaTeX => DVI} to \texttt{LaTeX => PDF}.
	\end{enumerate}

\section{Hello World!}
	\begin{enumerate}
		\item Create a new document.
		\item In the writing pane, write the following lines: 		
			\begin{lstlisting}
\documentclass{article}	
	\begin{document}				
		Hello World							
	\end{document}					
			\end{lstlisting}	
		\item Save the file.
		\item Click on the compile button.
		\item Open the \texttt{*.pdf} file.
	\end{enumerate}	
	
\section{Configure the Compiler Profile}
	To take advantage of the latest functionality, we need to change the compilers for our thesis document. 
	\begin{enumerate}
		\item In the top menu bar, navigate to \texttt{Build > Define Output Profiles}. 
		\item Select \texttt{LuaLaTeX} and select \texttt{Copy}. 
		\item Name it \texttt{LuaLaTeX to PDF (biber)}, and select the new profile you have created. 
		\item In the option marked \texttt{Path to Bibtex executable}, change the word \texttt{bibtex} to \texttt{biber}. 
		\item Exit this menu and select \texttt{LuaLaTeX => PDF (biber)} as you output profile in the drop-down menu. 
		\item Try compiling the University Thesis Document (press \texttt{F7}). It should compile without reporting any errors. 
	\end{enumerate}

\section{Customise the University Thesis Document}
	For the following exercises, recompile the document after you have completed each set of changes.
		\begin{enumerate}	
			\item Find and modify the following lines:
				\begin{lstlisting}
\begin{document}
...
\title{The University Thesis File}
\author{The Author's name}
% Faculty of Life Sciences people should comment the next line out
\school{The Author's school}
\faculty{The Author's faculty}
				\end{lstlisting}
			\item Modify the contents of the abstract (located between \lstinline|\beforeabstract| and \lstinline|\afterabstract|).
			\item Add some acknowledgements in the section immediately after \lstinline|\prefacesection{Acknowledgements}|.
		\end{enumerate}

\section{Adding a Chapter}
	\begin{enumerate}
		\item Create a new document.
		\item Type \lstinline|\chapter{Methodology}| in the writing pane to define it as the start of a chapter.
		\item Try some different writing commands, using the following examples:
			\begin{lstlisting}[upquote=true]
This is going to include some ``quoted text''.				
Some characters are reserved (i.e. they have a special purpose) but can be accessed by escaping with a backslash, e.g. \%.
To use italics, try \textit{using this command!}.
Math mode is contained within dollars, e.g. $\delta$.
			\end{lstlisting}
		\item Include this chapter file in the main \texttt{thesis.tex} after the other included \texttt{*.tex} file. If you saved your chapter file as `\texttt{mychapter.tex}', use the command \lstinline|\include{mychapter}|.
		\item Recompile the \texttt{thesis.tex} file.
	\end{enumerate}
	
\section{Referencing}
	\begin{enumerate}
		\item Open the thesis document and ensure the \lstinline|\bibliography{}| command points to the bibliography database (the \texttt{*.bib} file).
		\item Open your new chapter file, start a new paragraph (insert a blank line) then add some referencing commands using references from the bibliography file. Some examples are given below:
		\begin{lstlisting}
\citep{Payne2011}		
\citealp{Porinchu2003,Payne2011}	
\citet{Payne2011}			
\citealt{Payne2011}			
\citep[e.g.][]{Porinchu2003}		
\citep[][p.245]{Porinchu2003}	
		\end{lstlisting}
	\end{enumerate}

\section{Cross-referencing}
	\begin{enumerate}
		\item Add a reference label to the chapter, using the command \lstinline|\label{ch:mychapter}|.
		\item Cross reference the chapter, e.g. 
			\begin{lstlisting}
As described in chapter \ref{ch:mychapter}, this document is written in \LaTeX.
			\end{lstlisting}
		\item Remember you'll need to use a unique name inside the curly braces for each section, chapter, figure etc.
	\end{enumerate}
	
\section{Adding Figures}
	\begin{enumerate}
		\item Use the following code to add a figure:
			\begin{lstlisting}
\begin{figure}[htbp]							
	\centering								
		\includegraphics[width=1\textwidth]{example-image-a}
	\caption[Short caption]{Long caption}			
	\label{labeltexthere}						
\end{figure}
		\end{lstlisting}
		\item Try modifying the terms in the square brackets after \lstinline!\includegraphics!; experiment with \lstinline!\pageheight! or \lstinline!keepaspectratio!. Try using multiple terms separated with a comma (\lstinline!,!). 
		\item Now try the \lstinline!\sidewaysfigure! package. Remember you might need to add it to the preamble!
	\end{enumerate}
	
\section{Drawing Tables}
	\begin{enumerate}
		\item Create a table using the following lines of code:
			\begin{lstlisting}
\begin{table}[htbp]						
	\centering							
	\begin{tabular}{lcc}					
		\toprule						
			Name    & Age & Gender \\ \midrule	
			Tom     & 26  & Male   \\			
			James   & 22  & Male   \\			
			Eleanor & 25  & Female \\			
		\bottomrule					
	\end{tabular}						
	\caption[Short caption]{Long caption}		
	\label{}							
\end{table}
			\end{lstlisting}
		\item Try adding a column for attendance by adding \lstinline!c! to the \lstinline!\begin{tabular}{lcc}! and some data to each row.
		\item Remember cells are separated with an ampersand (\lstinline!&!), so you'll need to add this to each row.
	\end{enumerate}
	
\section{Writing Equations}
	\begin{enumerate}
		\item Create an equation using the following code as an example:
			\begin{lstlisting}
\begin{equation}			
	e=mc^{2}			
\label{massenergyequiv}	
\end{equation}
			\end{lstlisting}
		\item Reference the equation in the text using \lstinline!\ref{massenergyequiv}!.
	\end{enumerate}

\section{Inserting a quote}
	Try inserting a quote using the \lstinline!\quote! environment:
		\begin{lstlisting}
\begin{quote}
I think therefore I use \LaTeX for typesetting. 
\end{quote}
		\end{lstlisting}

\section{Escaped \& Special Characters}
	Try using reserved characters like \&, \textbackslash, and \%, and you'll run into difficulty. Try writing the following:
		\begin{lstlisting}
If you need to write an ampersand (\&), use the slash to escape or it will not render. 							
Similarly, if you wanted to use a slash or backslash (\backslash), you must use a command. If you need accents, they are often produced l\'ike that. 	
		\end{lstlisting}

\begin{figure}[b]
\doclicenseThis
\end{figure}

\end{document}